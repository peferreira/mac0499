%% ------------------------------------------------------------------------- %%
\chapter{Algoritmos}
\label{cap:algoritmos}

\paragraph*{}
Nesta se��o � explicado em mais detalhes o que � o SLIC e suas caracter�sticas. Logo em seguida, ser� descrito em mais detalhes o algoritmo da IFT, que ser� o algoritmo utilizado nos experimentos.

\section{SLIC - Simple Linear Iterative Clustering}
\paragraph*{}

Simple linear iterative clustering \citet{slic:sp_slic} � um m�todo para  gera��o de superpixels baseado  em gradientes ascendentes. Algoritmos dessa classe utilizam-se de m�ltiplas itera��es  para refinar um conjunto de pixels, at� que um crit�rio seja satisfeito, para a forma��o de um superpixel. 
\paragraph*{}
Esse algoritmo � mais r�pido, utiliza menos mem�ria, realiza �tima ader�ncia com bordas e  fronteiras de objetos na imagem e melhora a performance de algoritmos de segmenta��o. Simple linear iterative clustering � uma adapta��o do k-means para a gera��o de superpixels, com duas importantes distin��es:




\begin{itemize}
\item O n�mero de c�lculos de dist�ncia � drasticamente reduzido ao limitar o espa�o de busca para uma regi�o proporcional ao tamanho do superpixel. Isso faz com que a complexidade seja reduzida a linear no n�mero de pixels N e independente do n�mero de superpixels k. 

\item Uma medida ponderada de dist�ncia que combina cor e proximidade espacial, promovendo controle sobre o tamanho e a compacidade dos superpixels.

\end{itemize}
\pagebreak
\section{IFT - Image Foresting Transform}
\paragraph*{}
A Transformada Imagem-Floresta (IFT - Image Foresting Transform)  reduz problemas de processamento de imagem baseados em conexidade ao c�lculo de uma floresta de caminhos �timos no grafo derivado da imagem, seguido de um p�s-processamento simples de atributos da floresta resultante, como visto em  \citet{ift:ift_teoria}.
\paragraph*{}
 O Algoritmo da IFT pode ser usado para separar objetos do fundo em uma imagem. Nesse caso a IFT realiza uma busca atrav�s dos elementos da imagem, a partir de sementes iniciais que podem ser de objeto ou de fundo, por vizinhos similares. No decorrer da busca, elementos s�o conquistados virando objeto ou fundo e formando o conjunto final, que se trata de uma imagem bin�ria.

















 



